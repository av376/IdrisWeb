\section{Introduction}

Web applications, whilst ubiquitous, are also prone to incorrect construction
and security exploits such as SQL injection \cite{owasp:sqli} or cross-site
scripting \cite{owasp:xss}. Security breaches using such exploits are
far-reaching, and high profile cases involve large corporations such as Sony,
who suffered a well-publicised and extremely costly SQL Injection breach in
2011 \cite{ieee:sony}, and \textit{Yahoo!}, who suffered a breach in 2012
\cite{imperva:yahoo}. 

Many web applications are written in dynamically-checked scripting languages
such as PHP, Ruby or Python, which facilitate rapid development
\cite{w3techs:webpls}. A significant drawback, however, is that such languages
do not provide  the same static guarantees about runtime behaviour afforded by
programs with more expressive, static type systems, instead relying on
extensive unit testing to ensure correctness and security. 

%%------- EXAMPLE
Let us consider a simple database access routine, written in
PHP, where we wish to obtain the name and address of every employee working in
a given department, \texttt{\$dept}. We firstly construct an object
representing a database connection, where the arguments are the database host,
user, password and name respectively:


%
\begin{Verbatim}
  $conn = new mysqli("localhost", "username", 
                     "password", "db");
\end{Verbatim}
%
We then check to see if the connection was successful, and exit
if not.  This check is optional, so
it would be possible to omit it. However, this would cause
problems with later steps.

%
\begin{Verbatim}
  if (mysqli_connect_errno()) {
      exit();
  }
\end{Verbatim}
%
We then create a prepared statement detailing our query, and bind the
\texttt{`dept'} value:

\begin{Verbatim}
  $stmt = $conn->prepare("SELECT `name`, `address` 
     FROM `staff` WHERE `dept` = ?);
  $stmt->bind_param('s', $dept);
\end{Verbatim}
%
After the parameters have been bound, we execute the statement, assign
variables into which results will be stored, and fetch each row in turn. 
Failure to execute a statement before attempting to fetch rows would cause an error, as would attempting to execute a statement without binding variables to it.

%
\begin{Verbatim}
  $stmt->execute();
  $stmt->bind_result($name, $address);
  while ($stmt->fetch()) {
    printf("Name: %s, Age: %s", $name, $age);
  }
\end{Verbatim}
%
Finally, once the statement and connection are no longer needed, they should be
closed in order to discard the associated resources:

%
\begin{Verbatim}
  $stmt->close();
  $conn->close();
\end{Verbatim}
%
Even in this small example, there exists a specific resource usage protocol.
Firstly, a connection to the database must be opened. The object-oriented style
used in the example encapsulates this to an extent, as the object must be
created in order for operations to be performed, however it is less obvious in
a procedural version of the code. Secondly, a prepared statement is created,
using the raw SQL and placeholders to which variables are later bound. The
statement is then executed, and each row is retrieved from the database.
Finally, the resources are freed. 

Problems may arise if the protocol is not followed correctly.
A developer may, for example, accidentally close a statement whilst still
retrieving rows, which would cause a runtime error. Similarly, a programmer may
omit closing the statement or connection, which can lead to
problems such as resource leaks in longer-running server applications.
However, in conventional programming languages, there is no way to check
automatically that a protocol is followed.

In contrast, the use of \textit{dependent types} makes it possible
to specify a program's behaviour precisely, and to check that a 
specification is followed.
%
The difficulty is 
that automatic verification by a compiler can be difficult or
often impossible, requiring additional proofs to be given by the programmer.

This complexity can be ameliorated through the use of \textit{embedded
domain-specific languages} (EDSLs), which aim to abstract away the
complexity of the underlying type system. EDSLs allow domain experts to
write verified domain-specific code, with the EDSL itself providing the implicit proof that the written code is correct.

\idris{} \cite{brady2011idris} is a language with full dependent types, and
extensive support for EDSLs through overloading and syntax macros. Through the
use of \idris{}, and a framework for describing resource protocols using
\emph{algebraic effects}~\cite{brady:effects}, we
present a dependently-typed web framework, which allows the construction of
programs with additional guarantees about correctness and security, whilst
minimising the increase in development complexity. 

\subsection{Contributions}
The primary contribution of this paper is the application of 
dependent types to provide strong static guarantees
about the correctness and security of web applications, whilst minimising
additional development complexity. In particular, we present:

\begin{itemize}
\item A form-handling mechanism, which preserves type information and
manages user input, therefore
increasing applications' resilience to attacks such as SQL injection and cross-site scripting.

\item Representations of CGI, Databases and sessions as
\textit{resource-dependent algebraic effects}, allowing programs to be accepted only when they
follow clearly defined resource usage protocols.

\item A message board application, demonstrating the usage of the framework.

\end{itemize}

We structure the remainder of this paper as follows. We provide a brief
overview of the \texttt{Effects} framework in Section ~\ref{effects}; explain
how this may be used to ensure adherence to resource usage protocols for CGI,
SQLite and a session handler in Section ~\ref{rup}; describe an EDSL for
type-safe form handling in Section ~\ref{form}, implemented using
\texttt{Effects}; and discuss the larger example
of a message board system making use of these components in Section
~\ref{messageboard}.

The code used to implement the framework and all associated examples used in this paper is available online at \url{http://www.github.com/SimonJF/IdrisWeb}.
% =================================================
% =================================================

